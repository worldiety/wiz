% DO NOT MODIFY, GENERATED BY genformat.go
\section{On-disk format specification}
\subsection{Node}

Fields
\begin{table}[htp]
\centering
\begin{tabular}{l|l|l}
Id & Discriminator & \\ \hline
\end{tabular}
\end{table}
\subsection{VDevLabel}
A VDevLabel is the first node in the first block. It contains all information related to the entire vdev, like the block size. A VDevLabel is usually not updated and not part of any copy-on-write semantics. When updated, it is entirely overwritten. For recovery purposes the VDevLabel is repeated at block 16, if there are more than 16 blocks available.
Fields
\begin{table}[htp]
\centering
\begin{tabular}{l|l|l}
Magic & [8]byte{'w','i','z','b','l','o','c','k'} & The 8 magic bytes identifying a vdev block file.\\ \hline
Version & uint32{1} & The version is a 4 byte integer. Future versions are always backwards compatible and only if new nodes are introduced this number will be increased and get incompatible with older versions.\\ \hline
CustomMagic & [32]byte & The custom magic flag is an arbitrary defined 32 byte sequence which can be used to specify a sub format. This should be considered to be a string in a reverse domain name notation, like 'de.worldiety.cardiety.project\#\#\#'. You can send us your magic so that we keep a list of it.\\ \hline
PoolId & [32]byte & The unique 32 byte id of the pool in which this vdev is. Usually this Id should be generated with a secure random generator to ensure a high probability of a world wide uniqueness.\\ \hline
Id & uint32 & There may be 2\textasciicircum32 vdevs in a pool and each one must be unique to allow cross pointer references.\\ \hline
Blocksize & uint32 & An unsigned 4 byte integer denoting the block size in bytes for this vdev. By default this is 4096 and should not be smaller than 512 and not larger than 128KiB, but your mileage may vary. Always use power of 2 values.\\ \hline
HashAlgorithm & HashAlgorithm & Within a vdev the HashAlgorithm is a one-size-fits all option.\\ \hline
\end{tabular}
\end{table}
\subsection{HashAlgorithm}
A HashAlgorithm is always 32 byte long but may be calculated by different algorithms. You should always choose a cryptographically strong algorithm.
Fields
\begin{table}[htp]
\centering
\begin{tabular}{l|l|l}
\end{tabular}
\end{table}
\subsection{Hash}
A Hash is always 32 byte long but may be calculated by different algorithms.
Fields
\begin{table}[htp]
\centering
\begin{tabular}{l|l|l}
\end{tabular}
\end{table}
\subsection{TxReference}
A TxReference is used in the TxRing and refers to a transaction node.
Fields
\begin{table}[htp]
\centering
\begin{tabular}{l|l|l}
Ptr & NPtr & \\ \hline
Checksum & Hash('nptr'+Ptr) & The checksum of this node.\\ \hline
\end{tabular}
\end{table}
\subsection{NPtr}
A NPtr is a pointer to a node within a vdev. The block number is Offset % blocksize. It does not make sense to split the offset into a block number because 2\textasciicircum64 can point to an offset at 16 EiB. In contrast to that, separating that hypothetically e.g. into a 2\textasciicircum32 number for addressing a block number at 4k blocksize, we could only point to 16 TiB + offset. If we take all vdevs into account (2\textasciicircum32 * 2\textasciicircum64), which is effectively a 96 bit pointer, allows us to address 65.536 YiB. In BTRFS the pointer including the object id is even larger, but wastes that space. TODO think about making this a 12 byte array to save the 4 byte padding of the 4dev id. This could save us 25% of memory.
Fields
\begin{table}[htp]
\centering
\begin{tabular}{l|l|l}
VDev & uint32 & vdev denotes the id of the referenced vdev.\\ \hline
Offset & uint64 & The absolute offset of the node within a vdev. We do not address blocks directly.\\ \hline
\end{tabular}
\end{table}
\subsection{TxRing}
The TxRing data structure is a ring buffer which refers to the latest valid transaction roots. The entry with the highest transaction number and a valid checksum points to the root of the valid state of the storage.
Fields
\begin{table}[htp]
\centering
\begin{tabular}{l|l|l}
Transactions & [16]TxReference & \\ \hline
\end{tabular}
\end{table}
