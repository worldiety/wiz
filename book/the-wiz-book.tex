\documentclass[9pt,pagesize,DIV12,normalheadings,BCOR5mm,headexclude,footexclude]{scrbook} 
\setlength{\paperwidth}{170mm} 
\setlength{\paperheight}{220mm} 
\usepackage{calc} 
\typearea[current]{last} 
\usepackage{listings}
\usepackage{xcolor}
\usepackage[colorinlistoftodos,prependcaption,textsize=tiny]{todonotes}


\title{the wiz book -\\a complete guide to the wiz storage}
\date{December 27, 2018}
\author{Torben Schinke}


\begin{document}

\maketitle
\tableofcontents

\chapter{Preface}
The idea of a robust, simple and scalable storage format superseeding the 
lowest denomiator filesystems, fascinated me already 15 years ago, 
however I never had the opportunity to actually start implementing such a 
project. 
When the time came, I started to design a paper based specification in 2015 
which 
performs well for deduplicating large files, nested directory trees and 
continues snapshots. To solve the typical problems of a 'multi file based 
document format' at work, I created a proprietary java based implementation 
from it, called wiz - which is just the opposite of a git, similarities are 
purely coincidental. For the original intention, it worked pretty well. 
But as requirements changed, the performance for a lot of additional use 
cases was disappointing. The main performance issues are caused by both, 
inherent format decisions and the necessity of a complex virtual machine. 
In practice, the latter caused also penalties on the probably most successful 
mobile platform of our time. To solve all of these issues I started to design 
an entirely new specification which addresses all of the new additional 
scenarios (and even more). Hereafter this new specification is actually 
'wiz version 3' or simply 'wiz'. Therefore the proprietary existing wiz 
implementation is called 'legacy wiz' and is not only implemented in a 
different language but also does a lot of things differently to improve 
performance, storage usage, reliability and system complexity. 
Today, the market for closed source commercial software libraries is nearly 
dead and gaining money or finding acceptance is not easy. 
Usually large companies dominate the market with a lot (but definity not all) 
high quality products.

\chapter{Requirements}
In software development one distinguishes functional and non-functional 
needs. A functional requirement (FR) describes what a system is supposed to 
do on a certain input and involves typically some sort of calculation and 
processing. In contrast to that, non-functional requirements (NFR) define 
how a system behaves, e.g. in terms of response speed, memory consumption 
or security. Another important aspect of NFRs is software quality and 
maintainability. 

\section{functional requirements}
When following a comprehensive requirements analysis, one has to interview
the target group or rather the customer. The result is a list of rated
must- and should-be criteria. The following list is an opionated view of the
requirements as we have identified them and only include the must-have needs.

\subsection{FR-01: library}
The wiz storage format is an embedded database and must be includable into 
existing or new programs. The lowest common denomiator is a c-based ABI.
Even if there is nothing like a \textit{standard} ABI, each relevant 
operating system provides a c toolchain. Therefore the library must support
the c calling conventions for static or shared libraries for at least
the following architecture and operating system combinations:

\begin{table}[htp]
    \centering
    \begin{tabular}{l|l}
    \textbf{Operating System} & \textbf{Architecture}          \\ \hline
    Linux                     & x86\_32, x86\_64, armv7, armv8 \\ \hline
    Windows                   & x86\_32, x86\_64               \\ \hline
    MacOS                     & x86\_64                        \\ \hline
    Android                   & armv8                          \\ \hline
    iOS                       & armv8                         
    \end{tabular}
    \caption{At least supported platforms}
    \label{table:targets}
\end{table}

\subsection{FR-02: tooling}
The \textit{wiz} command line tool is available for all 
architecture and operating system combinations as defined 
in table \ref{table:targets}. This tool allows at least to create, 
modify and inspect \textit{wiz files} on a file
level, just like the unix \textit{tar} or \textit{zip} programs.

\todo{tbd}
The man page and command line interface.

\begin{lstlisting}[language=bash]
    $ wiz create storage.wiz
\end{lstlisting}

\subsection{FR-03: format}
The wiz storage format can also be called a \textit{repository}, a
\textit{filesystem} or a \textit{database}. These terms can be used
interchangeably. It supports the following modes of operation:

\begin{itemize}
    \item A single repository file on a conventional filesystem.
    \item A single repository file with one or multiple external log files.
    \item Multiple repository files with none, one or multiple external log 
    files.
    \item A raw disk file, with a maximum fixed size.
    \item A simple remote storage for repository files like a ftp server.
    These remotes do not necessarily support random access options or have
    other limitations like maximum file sizes, limited file name lengths,
    limited amount of entries per folder, no folder at all etc. 
    \item Read-only variants of all noted above.
\end{itemize}

\todo{tbd}
A cluster mode with a consensus algorithm. Note: RAFT is not a good one, 
because it fails to scale. Probably something simple, like sharding with
delayed replication. IMHO better to have something without guarantees here,
than a system which is irrecoverable broken after e.g. a split brain error.

\subsection{FR-04: transaction}
Everything in wiz is a transaction. These transactions are always
isolated and provide a high read throughput by using MVCC 
(Multiversion Concurrency Control) implementations.
A repository supports an arbritrary amount of sub volumes, sharing the 
available space of the entire storage. A sub volume may optionally 
support infinite snapshots, an infinite and 
cryptographically verifiable history and deduplication.

\todo{tbd}
Deduplication can be a hard thing and has multiple trade ofs. 
ZFS has only online-deduplication and requires 1GiB RAM per 1TiB of 
storage to perform well. BTRFS only has offline-dedup but is hard
to use and probably scales also badly.

\section{non-functional requirements}

\subsection{NFR-01: mount speed}
The time to open a storage must be constant, independent of how many entries
are contained or how large those entries are. This rule does not apply
if the repository is located on a storage system without random access,
like a simple ftp server.

\subsection{NFR-02: open files}
Open files are a very limited resource and are usually restricted to a few
hundred per process. The library and command line must be capable of handling
an arbritrary amount of open storage files by using a custom file handle 
pooling.

\subsection{NFR-03: memory consumption}
Repositories can grow very large, right into the range of hundreds of terabytes.
Also a process may open thousands of repositories at once. So the memory
consumption must be defacto constant independent of how many repositories are
open or how large a repository is or how large a single entry is. However 
there must be options to increase the memory consumption to trade of resources
against performance, to avoid hitting the I/O subsystem when required.


\end{document}