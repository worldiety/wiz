\documentclass[9pt,pagesize,DIV12,normalheadings,BCOR5mm,headexclude,footexclude]{scrbook} 
\setlength{\paperwidth}{170mm} 
\setlength{\paperheight}{220mm} 
\usepackage{calc} 
\typearea[current]{last} 
\usepackage{listings}
\usepackage{xcolor}
\usepackage[colorinlistoftodos,prependcaption,textsize=tiny]{todonotes}


\title{the wiz book -\\a complete guide to the wiz storage}
\date{December 27, 2018}
\author{Torben Schinke}

\newcommand{\wiz}{WIZ }

\begin{document}

\maketitle
\tableofcontents

\chapter{Preface}
The idea of a robust, simple and scalable storage format superseeding the 
lowest denomiator filesystems, fascinated me already 15 years ago, 
however I never had the opportunity to actually start implementing such a 
project. 
When the time came, I started to design a paper based specification in 2015 
which 
performs well for deduplicating large files, nested directory trees and 
continues snapshots. To solve the typical problems of a 'multi file based 
document format' at work, I created a proprietary java based implementation 
from it, called wiz - which is just the opposite of a git, similarities are 
purely coincidental. For the original intention, it worked pretty well. 
But as requirements changed, the performance for a lot of additional use 
cases was disappointing. The main performance issues are caused by both, 
inherent format decisions and the necessity of a complex virtual machine. 
In practice, the latter caused also penalties on the probably most successful 
mobile platform of our time. To solve all of these issues I started to design 
an entirely new specification which addresses all of the new additional 
scenarios (and even more). Hereafter this new specification is actually 
'wiz version 3' or simply 'wiz'. Therefore the proprietary existing wiz 
implementation is called 'legacy wiz' and is not only implemented in a 
different language but also does a lot of things differently to improve 
performance, storage usage, reliability and system complexity. 
Today, the market for closed source commercial software libraries is nearly 
dead and gaining money or finding acceptance is not easy. 
Usually large companies dominate the market with a lot (but definity not all) 
high quality products.

\chapter{Requirements}
In software development one distinguishes functional and non-functional 
needs. A functional requirement (FR) describes what a system is supposed to 
do on a certain input and involves typically some sort of calculation and 
processing. In contrast to that, non-functional requirements (NFR) define 
how a system behaves, e.g. in terms of response speed, memory consumption 
or security. Another important aspect of NFRs is software quality and 
maintainability. 

\section{functional requirements}
When following a comprehensive requirements analysis, one has to interview
the target group or rather the customer. The result is a list of rated
must- and should-be criteria. The following list is an opionated view of the
requirements as we have identified them and only include the must-have needs.

\subsection{FR-01: library}
The wiz storage format is an embedded database and must be includable into 
existing or new programs. The lowest common denomiator is a c-based ABI.
Even if there is nothing like a \textit{standard} ABI, each relevant 
operating system provides a c toolchain. Therefore the library must support
the c calling conventions for static or shared libraries for at least
the following architecture and operating system combinations:

\begin{table}[htp]
    \centering
    \begin{tabular}{l|l}
    \textbf{Operating System} & \textbf{Architecture}          \\ \hline
    Linux                     & x86\_32, x86\_64, armv7, armv8 \\ \hline
    Windows                   & x86\_32, x86\_64               \\ \hline
    MacOS                     & x86\_64                        \\ \hline
    Android                   & armv8                          \\ \hline
    iOS                       & armv8                         
    \end{tabular}
    \caption{At least supported platforms}
    \label{table:targets}
\end{table}

\subsection{FR-02: tooling}
The \textit{wiz} command line tool is available for all 
architecture and operating system combinations as defined 
in table \ref{table:targets}. This tool allows at least to create, 
modify and inspect \textit{wiz files} on a file
level, just like the unix \textit{tar} or \textit{zip} programs.

\todo{tbd}
The man page and command line interface.

\begin{lstlisting}[language=bash]
    $ wiz create storage.wiz
\end{lstlisting}

\subsection{FR-03: format}
The wiz storage format can also be called a \textit{repository}, a
\textit{filesystem} or a \textit{database}. These terms can be used
interchangeably. It supports the following modes of operation:

\begin{itemize}
    \item A single repository file on a conventional filesystem.
    \item A single repository file with one or multiple external log files.
    \item Multiple repository files with none, one or multiple external log 
    files.
    \item A raw disk file, with a maximum fixed size.
    \item A simple remote storage for repository files like a ftp server.
    These remotes do not necessarily support random access options or have
    other limitations like maximum file sizes, limited file name lengths,
    limited amount of entries per folder, no folder at all etc. 
    \item Read-only variants of all noted above.
\end{itemize}

\todo{tbd}
A cluster mode with a consensus algorithm. Note: RAFT is not a good one, 
because it fails to scale. Probably something simple, like sharding with
delayed replication. IMHO better to have something without guarantees here,
than a system which is irrecoverable broken after e.g. a split brain error.

\subsection{FR-04: transaction}
Everything in wiz is a transaction. These transactions are always
isolated and provide a high read throughput by using MVCC 
(Multiversion Concurrency Control) implementations.
A repository supports an arbritrary amount of sub volumes, sharing the 
available space of the entire storage. A sub volume may optionally 
support infinite snapshots, an infinite and 
cryptographically verifiable history and deduplication.

\todo{tbd}
Deduplication can be a hard thing and has multiple trade ofs. 
ZFS has only online-deduplication and requires 1GiB RAM per 1TiB of 
storage to perform well. BTRFS only has offline-dedup but is hard
to use and probably scales also badly memory wise.

\subsection{FR-05: record size}
The default record size for \wiz is 4096 bytes but a configurable record size is 
allowed. The size must be a power of 2 and the minimal size is 512 bytes. 
A small record size increases fragmentation but a larger record sizes 
has less overhead and less 
fragmentation at the cost of more memory and larger i/o operations. 
For example, if you request 16 KiB but your record size is 1MiB you will
saturate the bandwidth very early, however if you know that your payload
is always larger than 1MiB you use a record size of 512, you will probably 
saturate your IOPS earlier. Also records are loaded into memory, 
especially when dealing with concurrency, and therefore increase the memory
consumption.
Please note that the record size is fixed per pool, which means
that the record size cannot be changed after the creation of a repository.

\subsection{FR-06: resilience}
Resilience is a major design goal. The legacy \wiz has a packed format, using
a variable node size up to a configurable maximum size. This is basically a
good thing but has also the major drawbacks of an added overhead for length 
parsing, efficient buffer recycling and most importantly when recovering
from bit failures. As soon as a damage in the length header occurs, 
it is impossible to calculate the beginning of the next node. To mitigate this
issue one can insert boundary restart markers and checksums to skip 
broken parts and find valid points. However all of these workarounds increase
complexity and decrease performance. 
However, as in legacy \wiz a node may have a variable size but it always fits 
and aligns with a record. In a potential streaming scenario with an unknown 
offset, a boundary node can be inserted to detect the correct record offset.

The definition of a fixed record size and boundary nodes allows efficient
recovery mechanisms. When using nodes with checksums, it will be very simple
to restore nodes from an arbritrarily damaged storage. This certainly does
not mean that actual stored data can be read, especially if the data is 
fragmented accross multiple records. This can be improved a lot when using
nodes with a block chain data structure instead of using offset pointers.
In block chain data structures, the role of a pointer is replaced with 
a unique hash of the referenced content. Using this technique and saving
redundant nodes, makes it an easy task to recover 
a damaged storage.



\section{non-functional requirements}

\subsection{NFR-01: mount speed}
The time to open a storage must be constant, independent of how many entries
are contained or how large those entries are. This rule does not apply
if the repository is located on a storage system without random access,
like a simple ftp server.

\subsection{NFR-02: open files}
Open files are a very limited resource and are usually restricted to a few
hundred per process. The library and command line must be capable of handling
an arbritrary amount of open storage files by using a custom file handle 
pooling.

\subsection{NFR-03: memory consumption}
Repositories can grow very large, right into the range of hundreds of terabytes.
Also a process may open thousands of repositories at once. So the memory
consumption must be defacto constant independent of how many repositories are
open or how large a repository is or how large a single entry is. However 
there must be options to increase the memory consumption to trade of resources
against performance, to avoid hitting the I/O subsystem when required.



% DO NOT MODIFY, GENERATED BY genformat.go
\section{On-disk format specification}
\subsection{Node}

Fields
\begin{table}[htp]
\centering
\begin{tabular}{l|l|l}
Id & Discriminator & \\ \hline
\end{tabular}
\end{table}
\subsection{VDevLabel}
A VDevLabel is the first node in the first block. It contains all information related to the entire vdev, like the block size. A VDevLabel is usually not updated and not part of any copy-on-write semantics. When updated, it is entirely overwritten. For recovery purposes the VDevLabel is repeated at block 16, if there are more than 16 blocks available.
Fields
\begin{table}[htp]
\centering
\begin{tabular}{l|l|l}
Magic & [8]byte{'w','i','z','b','l','o','c','k'} & The 8 magic bytes identifying a vdev block file.\\ \hline
Version & uint32{1} & The version is a 4 byte integer. Future versions are always backwards compatible and only if new nodes are introduced this number will be increased and get incompatible with older versions.\\ \hline
CustomMagic & [32]byte & The custom magic flag is an arbitrary defined 32 byte sequence which can be used to specify a sub format. This should be considered to be a string in a reverse domain name notation, like 'de.worldiety.cardiety.project\#\#\#'. You can send us your magic so that we keep a list of it.\\ \hline
PoolId & [32]byte & The unique 32 byte id of the pool in which this vdev is. Usually this Id should be generated with a secure random generator to ensure a high probability of a world wide uniqueness.\\ \hline
Id & uint32 & There may be 2\textasciicircum32 vdevs in a pool and each one must be unique to allow cross pointer references.\\ \hline
Blocksize & uint32 & An unsigned 4 byte integer denoting the block size in bytes for this vdev. By default this is 4096 and should not be smaller than 512 and not larger than 128KiB, but your mileage may vary. Always use power of 2 values.\\ \hline
HashAlgorithm & HashAlgorithm & Within a vdev the HashAlgorithm is a one-size-fits all option.\\ \hline
\end{tabular}
\end{table}
\subsection{HashAlgorithm}
A HashAlgorithm is always 32 byte long but may be calculated by different algorithms. You should always choose a cryptographically strong algorithm.
Fields
\begin{table}[htp]
\centering
\begin{tabular}{l|l|l}
\end{tabular}
\end{table}
\subsection{Hash}
A Hash is always 32 byte long but may be calculated by different algorithms.
Fields
\begin{table}[htp]
\centering
\begin{tabular}{l|l|l}
\end{tabular}
\end{table}
\subsection{TxReference}
A TxReference is used in the TxRing and refers to a transaction node.
Fields
\begin{table}[htp]
\centering
\begin{tabular}{l|l|l}
Ptr & NPtr & \\ \hline
Checksum & Hash('nptr'+Ptr) & The checksum of this node.\\ \hline
\end{tabular}
\end{table}
\subsection{NPtr}
A NPtr is a pointer to a node within a vdev. The block number is Offset % blocksize. It does not make sense to split the offset into a block number because 2\textasciicircum64 can point to an offset at 16 EiB. In contrast to that, separating that hypothetically e.g. into a 2\textasciicircum32 number for addressing a block number at 4k blocksize, we could only point to 16 TiB + offset. If we take all vdevs into account (2\textasciicircum32 * 2\textasciicircum64), which is effectively a 96 bit pointer, allows us to address 65.536 YiB. In BTRFS the pointer including the object id is even larger, but wastes that space. TODO think about making this a 12 byte array to save the 4 byte padding of the 4dev id. This could save us 25% of memory.
Fields
\begin{table}[htp]
\centering
\begin{tabular}{l|l|l}
VDev & uint32 & vdev denotes the id of the referenced vdev.\\ \hline
Offset & uint64 & The absolute offset of the node within a vdev. We do not address blocks directly.\\ \hline
\end{tabular}
\end{table}
\subsection{TxRing}
The TxRing data structure is a ring buffer which refers to the latest valid transaction roots. The entry with the highest transaction number and a valid checksum points to the root of the valid state of the storage.
Fields
\begin{table}[htp]
\centering
\begin{tabular}{l|l|l}
Transactions & [16]TxReference & \\ \hline
\end{tabular}
\end{table}


\end{document}